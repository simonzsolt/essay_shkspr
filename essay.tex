
Shakespeare's works are probably one of the most studied and reflected texts in existence today. Countless interpretations, adaptations and all kinds of variants have been created from the works attributed to this name. They are regarded in such a high esteem, that often when demonstrating revolutionary and novel ideas where texts are involved, it is not suprusing if the chosen texts are from the Shakespeare corpus. 
Such an extraordinary example is the long-term data preservation project [CITE Goldman 2013], that encodes data into the DNA of bacteria, where they used---among other media---all of Shakespear's sonnets. With a shorter timespan in mind, another method of preservation is creating digital editions, idealy in an online environment. There is no shortage of Shakespeare themed websites on the internet, whether they are concerned with works of different editions and genres, performances of plays, linguistical attributes and curiosities\footnote{Take for example the many insult generators, or Ian Dosher's \textit{William Shakespeare's Star Wars}, which rewrites the saga as if a play from the time of Shakespeare.}; they greatly differ in their aim and audience. 
But what they do have in common is that they produce a large amount of content, adding volume to this far from simple corpus. Content with no clear source refrences and unnoted editing. Which begs the question: what exactly am I reading---or more precisely---what kind of Shakespeare am I reading? The same question can of course be asked for printed editions as well, and already a few exceedingly well edited series (take `The Arden Shakespeare` editions for example) have done a satisfying job of publishing these works. The online editions however, due to having a relativly short historical background---the first online critical editions where produced in the late 90's, early 2000's and are just gaining popularity---are in much less satisfying state. Also the vast number of possibilites, which this approach provides and the ways of utilizing these, gives way to countless more troubles and solutions, than what we would face in printed works. This paper is concerned with the incomplete review of a few notable online editions of Shakespear's plays, with special attention to digital critical editions available openly on the internet.
First a few honorable mentions, that don't really aim to please the researchers needs, but are persistent and serve as a good basis of comparison. The `Shakespeare Online` (\url{http://www.shakespeare-online.com/}) website describes itself as a `study guide`, and as such it does a pretty well. It has been around since 1999, and provides a lot of secondary materials besides the actual works. The works however have no source reference, the site does not give any actual guidance as to how we arrived at these specific (and modern) versions of texts. Within the texts the only structural information we gain is exactly the same we wouuld get from hardcopy versions, where the text is broken into three columns: the name or description of the speakers, the text, and line numbers.\footnote{At this point it is important to note the incoherence of the numbering in some cases. The first scene of the first act of \textit{Julius Caesar} has numbers appearing at uneven intervals, and seem to be off count. (\url{http://www.shakespeare-online.com/plays/julius_1_1.html})} The code consists of ordinary html markup (using tables and rows) without any metadata whatsoever indicating the nature of the content. One example of the conventional apparatus can be found reading the works published here. The site makes an attempt to add a few notes (mostly explaining archaic usage), but the annotation of these notes is done through the not so well-organized line numbers. Even in 1999 they could have probably used html anchors as footnotes.
This kind of editing gives no further help in better indexing the content and making it more available for search engines and repositories (such as libraries). Furthermore it does not promote the reuse and formal analasys of the texts, it's main only function is to be read at a computer screen. 
