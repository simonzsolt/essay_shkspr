
Shakespeare's works are probably one of the most studied and reflected texts in existence today. Countless interpretations, adaptations and all kinds of variants have been created from the works attributed to this name. They are regarded in such a high esteem, that often when demonstrating revolutionary and novel ideas where texts are involved, it is not suprusing if the chosen texts are from the Shakespeare corpus. 
Such an extraordinary example is the long-term data preservation project [CITE Goldman 2013], that encodes data into the DNA of bacteria, where they used---among other media---all of Shakespear's sonnets. With a shorter timespan in mind, another method of preservation is creating digital editions, idealy in an online environment. There is no shortage of Shakespeare themed websites on the internet, whether they are concerned with works of different editions and genres, performances of plays, linguistical attributes and curiosities\footnote{Take for example the many insult generators, or Ian Dosher's \textit{William Shakespeare's Star Wars}, which rewrites the saga as if a play from the time of Shakespeare.}; they greatly differ in their aim and audience. 
But what they do have in common is that they produce a large amount of content, adding volume to this far from simple corpus. Content with no clear source refrences and unnoted editing. Which begs the question: what exactly am I reading---or more precisely---what kind of Shakespeare am I reading? The same question can of course be asked for printed editions as well, and already a few exceedingly well edited series (take `The Arden Shakespeare` editions for example) have done a satisfying job of publishing these works. The online editions however, due to having a relativly short historical background---the first online critical editions where produced in the late 90's, early 2000's and are just gaining popularity---are in much less satisfying state. Also the vast number of possibilites, which this approach provides and the ways of utilizing these, gives way to countless more troubles and solutions, than what we would face in printed works. This paper is concerned with the incomplete review of a few notable online editions of Shakespear's plays, with special attention to digital critical editions available openly on the internet.
 
