
\documentclass{article}
 
\begin{document}


Shakespeare's works are probably one of the most studied and reflected texts in existence today. Countless interpretations, adaptations and all kinds of variants have been created from the works attributed to this name. They are regarded in such a high esteem, that often when demonstrating revolutionary and novel ideas where texts are involved, it is not suprusing if the chosen texts are from the Shakespeare corpus. 

Such an extraordinary example is the long-term data preservation project [CITE Goldman 2013], that encodes data into the DNA of bacteria, where they used---among other media---all of Shakespear's sonnets. With a shorter timespan in mind, another method of preservation is creating digital editions, idealy in an online environment. There is no shortage of Shakespeare themed websites on the internet, whether they are concerned with works of different editions and genres, performances of plays, linguistical attributes and curiosities\footnote{Take for example the many insult generators, or Ian Dosher's \textit{William Shakespeare's Star Wars}, which rewrites the saga as if a play from the time of Shakespeare.}; they greatly differ in their aim and audience. 

But what they do have in common is that they produce a large amount of content, adding volume to this far from simple corpus. Content with no clear source refrences and unnoted editing. Which begs the question: what exactly am I reading---or more precisely---what kind of Shakespeare am I reading? The same question can of course be asked for printed editions as well, and already a few exceedingly well edited series (take `The Arden Shakespeare` editions for example) have done a satisfying job of publishing these works. The online editions however, due to having a relativly short historical background---the first online critical editions where produced in the late 90's, early 2000's and are just gaining popularity---are in much less satisfying state. Also the vast number of possibilites, which this approach provides and the ways of utilizing these, gives way to countless more troubles and solutions, than what we would face in printed works. This paper is concerned with the incomplete review of a few notable online editions of Shakespear's plays, with special attention to digital critical editions available openly on the internet.

First a few honorable mentions, that don't really aim to please the researchers needs, but are persistent and serve as a good basis of comparison. The `Shakespeare Online` (\url{http://www.shakespeare-online.com/}) website describes itself as a `study guide`, and as such it does a pretty well. It has been around since 1999, and provides a lot of secondary materials besides the actual works. The works however have no source reference, the site does not give any actual guidance as to how we arrived at these specific (and modern) versions of texts. Within the texts the only structural information we gain is exactly the same we wouuld get from hardcopy versions, where the text is broken into three columns: the name or description of the speakers, the text, and line numbers.\footnote{At this point it is important to note the incoherence of the numbering in some cases. The first scene of the first act of \textit{Julius Caesar} has numbers appearing at uneven intervals, and seem to be off count. (\url{http://www.shakespeare-online.com/plays/julius_1_1.html})} The code consists of ordinary HTML markup (using tables and rows) without any metadata whatsoever indicating the nature of the content. One example of the conventional apparatus can be found reading the works published here. The site makes an attempt to add a few notes (mostly explaining archaic usage), but the annotation of these notes is done through the not so well-organized line numbers. Even in 1999 they could have probably used HTML [CODE anchor]s as footnotes.

This kind of editing gives no further help in better indexing the content and making it more available for search engines and repositories (such as libraries). Furthermore it does not promote the reuse and formal analasys of the texts, it's main only function is to be read at a computer screen.

The next item on our agenda is a similarly built website to the previous one, with the distinction that \url{http://shakespeare.mit.edu/} only provides the main texts of the works of Shakespeare, without any auxiliary reading. The modest appearance of the page does not quell the boastful impressum where the page introduces itself as the first edition of the complete works of Shakespeare on the Web, extant since 1993! 

The transcription of the plays originates from the `Moby project`\footnote{\url{http://icon.shef.ac.uk/Moby/}} under the name `Moby Shakespeare`\footnote{\url{http://icon.shef.ac.uk/Moby/mshak.html}}. The original file (which took up two, 1.4 MB floppy drivers) can be downloaded\footnote{\url{http://www.dcs.shef.ac.uk/research/ilash/Moby/mshak.tar.Z}} and read with a common text editor. This remnant of bygone era uses the very basic formatting options that a text editor can offer\footnote{Whenever I mention a text editor I make a distinction from a document editor, that offers rich formatting options. The likes we often refer to as \textit{text editors} today, such as MS Word, LibreOffice Writer and others.}, namely spaces, tabs and linebreaks with upper- and lowercase fonts. With such limitations it doesn't do a worse job of presenting the plays for reading as any of the latter websites do---and it does this on gracefully small disk space, which was still a relevant isse just a decade ago---appended with a glossary of terms used in the plays. But yet again, despite all of it's past glories it doesn't take much of a critical approach to the publishing of its contents, and the same can be said about the website that build on it.

But the website does introduce two or three (depending on the reading view) important features: line numbering and the annotation of speeches. By `reading view` I mean the option to choose between reading in segments of scenes. In this case the play is broken up into individual HTML files, the names of which correspond to the shorthand version of the title followed by the numbers of the act and scene, separated by `.`-s (e.g., [CODE allswell.1.3.html]). In addition to these files, there are seperate ones; concatenated versions of the segmeneted files. 

Let us first view the notation of the speeches in the segmentend versions. Seemingly they are non-existent and true enough they are not displayed on the pages. But if we take a deeper look into the source-code\footnote{The combination [CODE Ctrl+u] on most browsers should do the trick} we can squeeze out a bit of structural information regarding this text. Looking at the second scene of the first act in \textit{All's Well That Ends Well}\footnote{URL: \url{http://shakespeare.mit.edu/allswell/allswell.1.2.html}. Viewing the source in Chrome or Firefox: \url{view-source:http://shakespeare.mit.edu/allswell/allswell.1.2.html}.} the following can be said: The names of the actors are enclosed in [CODE anchor] tags ([CODE <a>]) with an attribe called `[CODE NAME]` wich takes values called `[CODE speechX]` where `[CODE X]` stands for a number. This number starts at [CODE 1] and increases every time someone speaks. The importance of this is first that every entity that performs an utterance can be distinguished from the actual utterances or any types of texts (as headers and stage directions). Secondly each utterance can be uniquely identified and put in sequence within the scope of the scene.  

Moving on from a speaker the actual utterance is broken up into lines enclosed in [CODE anchor] tags as well, and also equipped with the same attribute, but this it's values are purely numbers, starting from 1 and increasing untill the end of the scene. The lines of a specific utterance are wrapped into a [CODE blockquote] tag to form some sort of unity between the lines. Thus every line can be referenced the same as the speakers.

For example if I were to cite the line: `They that least lend it you shall lack you first` from this part based on the above I could do it precisely in the following form: the URL points uniquely to the specific act and play at hand (\url{http://shakespeare.mit.edu/allswell/allswell.1.2.html}), and viewing the source we find that this line is enclosed in an [CODE anchor] tag with a value of 76 to its [CODE NAME] attribute.\footnote{A theoretical syntax could look something like this: [CODE http://shakespeare.mit.edu/allswell/allswell.1.2.html:NAME=76].} 

Wrapping the lines of speech together is an important gesture, but in this case not sufficent to properly appoint the location of the group within the text or its relation to speaker. It is `merely` assumed by the reader that the lines following the name of the speaker are that of the speaker's, and furthermore those succeeding lines are related to each other. This of course sounds as trivial observation but it is exactly these trivialities that need to expressed in a computer environment for it to be more machine-readable.\footnote{To this matter a very fitting quote is from [CITE Kent 2012:xxii] on the topic of complexity in computer programing Kent says: `The thing that makes computers so hard to deal with is not their complexity, but their utter simplicity. The first thing that ought to be explained to the general public is that a computer possesses incredibly little ordinary intelligence. The real mystique behind computers is how anybody can manage to get such elaborate behavior out of such a limited set of basic capabilities. The art of computer programming is somewhat like the art of getting an imbecile to play bridge or to fill out his tax returns by himself. It can be done, provided you know how to exploit the imbecile’s limited talents, and are willing to have enormous patience with his inability to make the most trivial common sense decisions on his own. Imagine, for example, that he only understood grammatically perfect sentences, and couldn’t make the slightest allowance for colloquialisms, or for the normal way people restart sentences in mid-speech, or for the trivial typographical errors which we correct so automatically that we don’t even see them. The first step toward understanding computers is an appreciation of their simplicity, not their complexity`.}

Now briefly looking at the concatenated version (\url{http://shakespeare.mit.edu/allswell/full.html}) we can see in the source that the line numbering changes accordingly to the scope of the content. The act and scene numbers seen in the filenames before now appear in the `[CODE NAME]` attribute of each line. Thus the previous example would look like this: [CODE http://shakespeare.mit.edu/allswell/full.html:NAME=1.2.76]. Alas the notation of the speakers does not follow this logic and remains the same, therefore muddling up any bases of identification, where the attribute value of `speech14` is assigned to multiple speakers in several occasions.

All in all this online demonstrates some of the fundamental requirements and challenges that digital editions face, most of which are very much relevant to this day. Beyond the above shown structural editing and the reading of the texts the website doesn't offer any more capabilities than the previouse one. It is also a monument for a once respectable achivement. That being said there are traces of other features of the website in the `news` section (\url{http://shakespeare.mit.edu/news.html}), which also says a lot about how harsh the life of a website can be.   

`The Internet Shakespeare Editions` (\url{http://internetshakespeare.uvic.ca/}) is maintained by British Columbia's University of Victoria. The project dates back to 1991 but the it seems to be stagnant since 2013. The aim of the collection is clearly stated in the brief history of the site: `Our aim is to provide our world-wide audience with the best in Shakespeare scholarship and performance on a site that is freely available`.\footnote{\url{http://internetshakespeare.uvic.ca/Foyer/history/}}. Furthermore in describing the editing guidlines we learn that the texts are: `annotated to a deep scholarly level`.\footnote{\url{http://internetshakespeare.uvic.ca/Foyer/makingwaves/about2/}} There is also some mention about using XML markup, but they are not available anywhere nor it is apparent under what guidelines they are made, so what we are left is HTML rendered view again. And that does not reflect the the ambitions above.

The sources of the plays do seem to ad more detail to the code but these do not provide substantially more information than does the edition based on the `Moby Shakespeare` project. A positive quality of the page is  that most of plays are available in two versions: 17\supertext{th} century editions and modern transcription. The gidelines also give some information how these modern transcriptions come to be.\footnote{\url{http://internetshakespeare.uvic.ca/Foyer/guidelines/1-Introduction/}} The modern texts feature notes on specific parts of the texts, but the nature of these are mostly explanatory reading guides. Neither versions of the text show any philological notes. The line numbering is jumbled up in all versions.
A final point of interest about this edition is the scanned images of several early prints. Altough the texts and images are not connected in any way, and not even the most basic metadata\footnote{For example the fifteen basic elements set down by `The Dublin Core Metadata Initiative` (\url{http://dublincore.org/documents/dces/}).} are absent from the markup.

In conclusion the `The Internet Shakespeare Editions` gives widest variety of texts of the three editions so far, with multiple versions of the same text and scans of early prints which makes it a precious collection, but it certainly does not measure up to the standards of scholary digital editions of the recent years.

A much better application of bringing together text versions with thei respective images can be found at the `Shakespeare Electronic Archive` (\url{http://shea.mit.edu/shakespeare/htdocs/welcome/welcome.htm}).

As the homepage states, the site is in a test phase and it aims to combine the images of the `First Folio` with texts based on the `Oxford Text Archive` and using the trascriptions of `The Oxford Electronic Edition` for the `Quarto` editions. Both transcriptions date back to the early 90's.\footnote{In detail: \url{http://shea.mit.edu/shakespeare/htdocs/materials/materials.htm}.} 

The line numbering is finally consistent---this is achieved by simply numbering all the thrououg the play, even the stage instructions---and the lines of text are well synched with the lines on the scanned images. The images also have a basic set of metadata next to them, described here: \url{http://shea.mit.edu/shakespeare/htdocs/materials/html/contents.htm#FDF}.

This project---though incomplete---is a very promising one, solving problems connecting text and their images much better than previous examples.

The next item on this agenda is the `Open Source Shakespeare` (\url{http://www.opensourceshakespeare.org/}). The text publish here do not add more to the encoding than what we have seen before, it's a simple well formated, but still HTML edition. What makes it interesting is the wide range of textual analytical tools it provides. The most notable probably the advanced search options, which allows for several criteria (works, genres, charachters, time period) and format to be met in querying the texts. The statistics page (\url{http://www.opensourceshakespeare.org/stats/}) give a list of word frequencies, speech frequencies and lengths sorted by genres or characters. These are the main aspects in stylometric studies. The concordance (\url{http://www.opensourceshakespeare.org/concordance/}) gives a full view of word frequency in the texts. Perhaps a lemmatized list would be easier to browse. Lastly the character search function is very handy. Every speech of a character can be listed and searched, and the number of speeches per character are displayed to see how talkative they are.

These handy tools shown by OSS are based on a well organized database, but the texts do not comform with the needs of modern digital critical editions.

The purpose of examining these nuances in detail was to pave the road to the methods that aim to solve these problems of digital editions on the internet. The \textit{Text Encoding Initiative} (TEI) is slowly becoming the number one method for encoding text in scholary cirles.\footnote{The TEI guidelines (\url{http://www.tei-c.org/index.xml}) describe a recomended and commonly excepted list of expressions to encode text on the internet (most comfortably, but not exclusivly in XML format). The aim of this vocabulary is to promote well structured machine-readable texts and collaboration between different projects. The vocabulary is far from exhaustive (and it doesn't aim to be), but attempts to grasp the general ascpects of a specific type of text. It is up to researchers and editors to best represent their notions of their texts with this limited toolkit. In this paper the most relevant part of these guidlines will be the module of `Performance Texts` (\url{http://www.tei-c.org/release/doc/tei-p5-doc/en/html/DR.html})} In the following part I will revirew a few editions that use TEI guidelines and also publish the XML files.

First off the `Shakespeare Quartos Archive` (\url{http://www.quartos.org/index.html}) list thirty-two quarto copies of \textit{Hamlet} with their digital scans and transcriptions. These XML files consist over ten-thousand lines of code, therefore I will not engage in a detailed description of them, rather highligth the general idea of representation and compare them to the HTML based editions introduced previously.

\end{document}